% This file contains macros that can be called up from connected TeX files
% It helps to summarise repeated code, e.g. figure insertion (see below).

% insert a centered figure with caption and description
% parameters 1:filename, 2:title, 3:description and label
\newcommand{\figuremacro}[3]{
    \begin{figure}[t]
    \centering
    \includegraphics[width=1\linewidth]{#1}
    \caption[#2]{\textbf{#2} #3}
    \label{#1}
    \end{figure}
}

\newcommand{\figuremacroW}[4]{
	\begin{figure}[htbp]
		\centering
		\includegraphics[width=#4\linewidth]{#1}
		\caption[#2]{\textbf{#2} - #3}
		\label{#1}
	\end{figure}
}

\newcommand{\figuremacroPos}[4]{
    \begin{figure}[#4]
    \centering
    \includegraphics[width=.5\linewidth]{#1}
    \caption[#2]{\textbf{#2} #3}
    \label{#1}
    \end{figure}
}

\newcommand{\figuremacroAll}[3]{
    \begin{figure*}[!t]
    \centering
    \includegraphics[width=1\textwidth]{#1}
    \caption[#2]{\textbf{#2} #3}
    \label{#1}
    \end{figure*}
}

\newcommand{\figuremacroWR}[5]{
	\begin{figure}[htbp]
		\centering
		\includegraphics[width=#4\linewidth, angle=#5]{#1}
		\caption[#2]{\textbf{#2} - #3}
		\label{#1}
	\end{figure}
}

\newcommand{\figuremacroAllWR}[5]{
	\begin{figure*}[htbp]
		\centering
		\includegraphics[width=#4\textwidth, angle=#5]{#1}
		\caption[#2]{\textbf{#2} - #3}
		\label{#1}
	\end{figure*}
}

%% Example of use of subfiguremacro:
%\begin{figure}
%  \subfiguremacro{figure_1}{}{.5}
%  \subfiguremacro{figure_2}{}{.5}
%  \caption{}
%\end{figure}
\newcommand{\subfiguremacro}[3] {
    \begin{subfigure}[h]{#3\linewidth}
        \centering
        \includegraphics[width=1\linewidth]{#1}
        \caption{#2}
        \label{#1}
    \end{subfigure}
}
